This thesis will explore the notion of \textit{pairwise preference}, a concept with applications in political science, economics, social science, and machine learning (\cite{jordan}, \cite{arrow}).
A pairwise preference is simply \textit{a preference for one item over another, given two items.}
We will show that this representation is highly general, and that problems posed in this framework are amenable to many kinds of analysis.
More importantly, we believe that such a representation may be capable of representing human subjectivity accurately and precisely.

\bigskip

The choice to explore this particular representation was not arbitrary.
In the \textbf{historical context} section, we will review aspects of the history of science, philosophy, and politics, and attempt to discern some key themes.
This section will motivate the investigation of pairwise preference and predict some desirable theoretical properties.

In the \textbf{definitions} section, we will present the basic elements of socrata graphs.
We will show how a large class of preference resolution problems can be set up within this general framework.

In \textbf{applications}, we will demonstrate various algorithms for analyzing these types of graphs, and discuss their strengths and limitations.

In \textbf{future directions}, we will identify additional avenues of exploration. There are particularly interesting possibilities involving blockchain-based virtual machine (BBVM) technologies like Ethereum.