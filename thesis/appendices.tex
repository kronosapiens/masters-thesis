\subsection{Details of MMSB Derivation}
\label{sec:mmsb_appendix}

\subsubsection{Joint probability and ELBO}

Here is the joint probability for this model:
\begin{eqnarray*}
&p(Y, \pi_{1:P}, Z_{p \rightarrow q}, Z_{q \rightarrow p} | B, \alpha) = \\
&\prod_{n=1}^N p(y_n | z_{p_n \rightarrow q_n}, z_{q_n \rightarrow p_n} , B)
p(z_{p_n \rightarrow q_n} | \pi_{p_n}) p(z_{q_n \rightarrow p_n} | \pi_{q_n})
\prod_{p = 1}^V p(\pi_p| \alpha)
\end{eqnarray*}

Inference will involve learning posterior values for $\pi_{1:P}, Z_{p \rightarrow q}, Z_{q \rightarrow p}, B$.
We will learn $\pi_{1:P}, Z_{p \rightarrow q}, Z_{q \rightarrow p}$ through variational inference, and $B$ through variational expectation-maximization (as it is not a random variable).

We introduce the following $	q$ distributions for the latent variables:
\begin{eqnarray*}
q(\pi_p) &\sim& Dirichlet(\gamma_p) \\ 
q(z_{p_n \rightarrow q_n}) &\sim& Multinomial(\phi_{p_n \rightarrow q_n}) \\
q(z_{q_n \rightarrow p_n}) &\sim& Multinomial(\phi_{q_n \rightarrow p_n}) 
\end{eqnarray*}

Note that the matrix $\Gamma$ will be $V \times K$, while matrices $\Phi_{p \rightarrow q}$ and $\Phi_{q \rightarrow p}$ are $N \times K$.
We will learn all parameters by maximizing the ELBO:
\begin{eqnarray*}
&ELBO&(\Gamma,\Phi_{p \rightarrow q},\Phi_{q \rightarrow p};Y, \pi_{1:P}, Z_{p \rightarrow q}, Z_{q \rightarrow p},B,\alpha) = \\
&\mathbb{E}_q&\bigg[\sum_{n=1}^N \bigg(\log p(y_n | z_{p_n \rightarrow q_n}, z_{q_n \rightarrow p_n} , B)
+ \log p(z_{p_n \rightarrow q_n} | \pi_{p_n})
+ \log p(z_{q_n \rightarrow p_n} | \pi_{q_n})\bigg) \\
&&+ \sum_{p = 1}^V \log p(\pi_p| \alpha)\bigg] \\
- &\mathbb{E}_q&\bigg[\sum_{n=1}^N \bigg( \log q(z_{p_n \rightarrow q_n} | \phi_{p_n \rightarrow q_n})
+ \log q(z_{q_n \rightarrow p_n} | \phi_{q_n \rightarrow p_n})\bigg) 
+ \sum_{p = 1}^V \log q(\pi_p| \gamma_p)\bigg]
\end{eqnarray*}

The updates for $\gamma, \phi_{p \rightarrow q}, \phi_{q \rightarrow p}$ are exactly as given in \cite{airoldi:2008}, with the modification that we iterate over $N$ observations, rather than $V \times V$ pairs.

\subsubsection{Learning the $B$ Matrix}

Our model differs from the MMSB specified by \cite{airoldi:2008} , in that we introduce restrictions on the matrix B.
The first thing to note is that the symmetric restriction on B implies that we must learn and store only the upper-triangle of the matrix; the lower-triangle can be generated from the upper.
We formalize this symmetry with the following likelihood distribution on $y_n$ (with $g$ referring to the index corresponding to the one-hot vector $z_{p_n \rightarrow q_n}$, and $h$ corresponding to the same for $z_{q_n \rightarrow p_n}$):
\begin{eqnarray*}
&& p(y_n | z_{p_n \rightarrow q_n} = g, z_{q_n \rightarrow p_n} = h, B) \\
&=& p(y_n | B_{gh})^{\mathbbm{1}[g < h]}p(y_n | 1 - B_{hg})^{\mathbbm{1}[g \geq h]} \\
&=& \bigg(B_{gh}^{y_n}(1-B_{gh})^{(1-y_n)}\bigg)^{\mathbbm{1}[g < h]} \bigg((1-B_{hg})^{y_n}B_{hg}^{(1-y_n)}\bigg)^{\mathbbm{1}[g \geq h]}
\end{eqnarray*}

Values of $B$ are learned through variational EM, in which we set the values using maximum-likelihood, using the values of the variational parameters learned during CAVI.
To derive the update for $B$, we take the gradient of the ELBO with respect to $B$, set it to $0$, and solve.
Note that we only need to consider the terms in the ELBO with depend on $B$; we hide all other terms in the constant $C$.
Additionally, we let $g_n = z_{p_n \rightarrow q_n}$, and $h_n = z_{q_n \rightarrow p_n}$:

\begin{eqnarray*}
ELBO(B) &=& C + \mathbb{E}_q\bigg[\sum_{n=1}^N \log p(y_n | z_{p_n \rightarrow q_n}, z_{q_n \rightarrow p_n} , B)\bigg] \\
&=& C+ \mathbb{E}_q\bigg[\sum_{n=1}^N \mathbbm{1}[g_n < h_n]\bigg(y_n \log(B_{g_n h_n}) + (1-y_n) \log(1-B_{g_n h_n})\bigg) \\
&&+ \mathbbm{1}[g_n \geq h_n]\bigg(y_n \log(1-B_{h_n g_n}) + (1-y_n) \log(B_{h_n g_n})\bigg) \bigg] \\
&=& C + \sum_{n=1}^N \bigg[ \sum_{g_n < h_n}\bigg( p(g_n)p(h_n) \big(y_n \log(B_{g_n h_n}) + (1-y_n) \log(1-B_{g_n h_n})\big) \bigg) \\
&&+  \sum_{g_n \geq h_n}\bigg( p(g_n)p(h_n) \big( y_n \log(1-B_{h_n g_n}) + (1-y_n) \log(B_{h_n g_n}) \big) \bigg) \bigg] \\
&=& C + \sum_{n=1}^N \bigg[ \sum_{g_n < h_n}\bigg( \phi_{p_n \rightarrow q_n, g_n} \phi_{q_n \rightarrow p_n, h_n} \big(y_n \log(B_{g_n h_n}) + (1-y_n) \log(1-B_{g_n h_n})\big) \bigg) \\
&&+  \sum_{g_n \geq h_n}\bigg( \phi_{p_n \rightarrow q_n, g_n} \phi_{q_n \rightarrow p_n, h_n} \big( y_n \log(1-B_{h_n g_n}) + (1-y_n) \log(B_{h_n g_n}) \big) \bigg) \bigg] 
\end{eqnarray*}

We now take the derivative with respect to $B_{gh}$, assuming that $g < h$:
\begin{eqnarray*}
\frac{\partial ELBO}{\partial B_{gh}} &=& \sum_{n=1}^N \bigg[ 
\phi_{p_n \rightarrow q_n, g} \phi_{q_n \rightarrow p_n, h} \bigg(\frac{y_n}{B_{gh}} - \frac{1-y_n}{1-B_{gh}}\bigg) \\
&+& \phi_{p_n \rightarrow q_n, h} \phi_{q_n \rightarrow p_n, g} \bigg(\frac{-y_n}{1-B_{gh}} + \frac{1-y_n}{B_{gh}}\bigg) \bigg]
\end{eqnarray*}

Setting this expression to $0$, and solving for $B_{gh}$, gives the following closed-form solution:
\begin{eqnarray*}
\hat{B}_{gh} = \frac{\sum_{n=1}^N \phi_{p_n \rightarrow q_n, g} \phi_{q_n \rightarrow p_n, h} y_n + \phi_{p_n \rightarrow q_n, h} \phi_{q_n \rightarrow p_n, g}(1-y_n)}{\sum_{n=1}^N \phi_{p_n \rightarrow q_n, g} \phi_{q_n \rightarrow p_n, h} + \phi_{p_n \rightarrow q_n, h} \phi_{q_n \rightarrow p_n, g}}
\end{eqnarray*}

\subsection{Datasets}
\label{sec:datasets}

\subsubsection{MovieLens-100k}
We now describe the way we generate pairwise preference data from the 
MovieLens-100k dataset.  At a high level, if a user $u$ rated $M_u$ movies,
we randomly sample $10*M_u$ pairs of movies rated by that user.  For each
randomly sampled pair, if the user gave different ratings to the 2 movies,
we add the user's pairwise preference to our dataset.  The exact algorithm
is shown in below.
Importantly, we run the above process for both the training data and the test
data.  We use the file `u5.base' as our training set, and `u5.test' as our test
set, from the MovieLens-100k dataset, which we downloaded at 
\url{http://grouplens.org/datasets/movielens/100k/}.  
There are 80,000 ratings in this training
set, and 20,000 ratings in this test set.
After running the above procedure on the training set, we separate $20\%$ of the
data as heldout, and use the rest for training.

\subsubsection{Data-generating Algorithm for MovieLens data}

For a given user $u$, let $M_u$ denote the set of movies rated by this user, and let
$r_{um}$ denote the rating user $u$ gave to movie $m$.

\begin{itemize}
	\item Let $D = \emptyset$, $n = 1$.
	\item For each user $u \in U$
	\begin{itemize}
		\item For $n \in $\{1, ..., $10\cdot|M_u|\}$
		\begin{itemize}
			\item Randomly select $p_n, q_n \in M_u$, where $p_n < q_n$.
			\item Let $x_n = (p_n, q_n, u_n)$, and $y_n = \mathbbm{1}  \left[r_{up_n} < r_{uq_n} \right]$
			\item If $(x_n,y_n) \notin D$ and $r_{up_n} \neq r_{uq_n}$
			\begin{itemize}
				\item $D = D \cup (x_n,y_n)$.
				\item $n=n+1$.
			\end{itemize}
		\end{itemize}
	\end{itemize}
\end{itemize}

\subsubsection{All Our Ideas}
All Our Ideas is an online platform enabling for the creation and distribution of ``wiki surveys'' (\cite{salganik:2015}) --- in which users are prompted to make pairwise preferences with items drawn from a dynamic answer pool.

As an academic project, All Our Ideas makes raw survey data available to download.
These data include information about the candidate answers and every comparison made, giving hashed IP addresses to preserve user anonymity.

This survey we consider, which ran over the summer of 2012, was used to answer a critical question: what beer to serve at the author's undergraduate going-away party.
The survey considered 27 beers and had 1468 responses from 17 IP addresses, with a majority of responses (1244) coming from a single IP address (not the author's).

