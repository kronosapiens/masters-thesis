\subsection{Deployment to a BBVM Environment}

\subsection{Active Learning and Item Pruning}

\subsection{Optimal Committee Discovery}

\subsection{Comparison to alternative representations}

The problem of representing subjectivity is not new; rather, it represents one of the fundamental challenges of the social sciences. It will be illustrative to compare this proposed system to common standards.

A \textit{Likert scale} is a type of survey, consisting of a number of \textit{Likert items}, each asking the participant to answer a question by checking boxes such as \say{Strongly Agree}, \say{Strongly Disagree} and so on. This method attempts to project subjectivity onto a range, for later interpretation via a number line.

In well-designed Likert scales, the responses distribute uniformly across the range, which can consist of any number of degrees of feeling (five to ten being common). This distribution of answers allows a researcher to justify interpreting the responses as though they fell along some sort of number line, and perform analysis accordingly.

The use of socrata avoids some of the problems observed with Likert scales. 

GENERALIZE TO LIKERT SCALE?

1 -> 2

1 -> 3

2 -> 3

4 -> 3

5 -> 3

5 -> 4


An important consideration is the treatment of contradiction (specifically, intransitivity) among preferences. In the political science literature, intransitivity is seen as problematic \cite{arrow}. We, however, view this type of contradiction as a property of subjective experience and do not attempt to eliminate it. Rather, we allow for this possibility and seek to interpret it.

\subsection{Relations to other branches of social science}

- Study of social influences on voting (music preference study)
The preferences measured via this approach will likely exhibit the same phenomenon as studied by.

There are many possible extensions of this theory.

Item clustering for algorithmic efficiency

Recursive decomposition of participants

Studying how question phrasing affects answers. By using the same answer set $A$ for multiple questions, and studying how the responses differ, it should be possible to gain a deeper understanding about the relationship between the questions.

Beyond economics, political theorist Hannah Arendt has written about the need for a \say{public sphere}, in which there exist methods and structures to allow the achievement of collective freedom via the construction of a common world. CITE.