
Ultimately, our goal is to make statements about the \textit{global} relationships which exist in the graph, using information which comes to us via \textit{local} relationships between nodes.

\subsection{Preference Resolution from Local Relations}

If a graph is not fully connected (some vertices lack edges), then it is possible for there to exist multiple sinks.
If a graph is fully connected, however, then there can exist at most one sink (if there were two sinks, one must point towards the other: a contradiction $\rightarrow\leftarrow$).

The first example we will consider is the most fundamental of preference resolution systems: the \textbf{plurality voting system}.
In this system, entities (which we will call voters) are allowed to vote for one of $n$ candidates, and the candidate with the most votes win.

To set up this system in the language of socrata graphs, we will define the following:
Our \textit{access policy} is that every voter is allowed to cast one vote in the election.
Each vote will be translated into $n-1$ socrata, with an edge pointing to the chosen candidate from every other candidate.
For example, given three candidates $\{a, b, c\}$, a vote for $a$ would translate into the following:

\begin{center}
	
\socrata{a}{b}

\socrata{a}{c}
	
\end{center}

We aggregate the socrata using the additive rule described above.
The winner of the election will be the sink of the resulting graph.
The presence of at least one sink is guaranteed by the structure of the access policy.
Consider the following result: $a$ receives 10 votes, $b$ receives 7, and $c$ receives 3.
We combine these into the following graph:

\begin{center}
\begin{tikzpicture}[node distance=3cm]
\node [roundnode] (a) {a};
\node [roundnode] (b) [right of=a] {b};
\node [roundnode] (c) [below of=a] {c};
\draw[ultra thick, -] (b) -- node[auto] {10 - 7} (a);
\draw[ultra thick, -] (c) -- node[auto] {10 - 3} (a);
\draw[ultra thick, -] (c) -- node[auto] {7 - 3} (b);
\end{tikzpicture}
\end{center}

Then taking the differences and creating the directed edges:

\begin{center}
\begin{tikzpicture}[node distance=3cm]
\node [roundnode] (a) {a};
\node [roundnode] (b) [right of=a] {b};
\node [roundnode] (c) [below of=a] {c};
\draw[ultra thick, ->] (b) -- node[auto] {3} (a);
\draw[ultra thick, ->] (c) -- node[auto] {7} (a);
\draw[ultra thick, ->] (c) -- node[auto] {4} (b);
\end{tikzpicture}
\end{center}

We run the sink-finding algorithm and return $a$, the winner.
It is worth nothing how the complexity of this preference-resolution scheme is $O(n)$, the same as the complexity of simply taking the maximum of vote counts for $n$ candidates.

What happens in the case of a tie?
Let's say $b$ receives 10 votes:

\begin{center}
\begin{tikzpicture}[node distance=3cm]
\node [roundnode] (a) {a};
\node [roundnode] (b) [right of=a] {b};
\node [roundnode] (c) [below of=a] {c};
\draw[ultra thick, -] (b) -- node[auto] {10 - 10} (a);
\draw[ultra thick, -] (c) -- node[auto] {10 - 3} (a);
\draw[ultra thick, -] (c) -- node[auto] {10 - 3} (b);
\end{tikzpicture}
\end{center}

\begin{center}
\begin{tikzpicture}[node distance=3cm]
\node [roundnode] (a) {a};
\node [roundnode] (b) [right of=a] {b};
\node [roundnode] (c) [below of=a] {c};
\draw[ultra thick, ->] (c) -- node[auto] {7} (a);
\draw[ultra thick, ->] (c) -- node[auto] {7} (b);
\end{tikzpicture}
\end{center}

In this case, we will observe two sinks in the resulting graph, representing two possible winners.
This corresponds nicely with our intuition of what should happen in this case.

\subsection{Preference Resolution from Global Relations}

Earlier, we showed how the preference resolution problem can be as simple as finding a sink in a directed graph, an $O(n)$ operation.
Without a guarantee of acyclicity, however, we must turn to alternative methods.
Intuitively, we would like to say that a node which has a lot of incoming edges is preferred, even if that node possesses some number of outgoing edges.

The simplest approach in this vein would be to take the difference of incoming and outgoing edges, and return the node with the largest difference.
This algorithm, which runs in $O(m + n)$ time, captures local information between nodes, but incorporates no information about global relationships among nodes.

\bigskip

Google's first \textbf{PageRank} algorithm, designed by founders Sergey Brin and Larry Page, was designed to solve a similar problem \cite{brin}: given a directed graph of websites, can one determine which sites are most relevant for a given query?
Brin and Page's solution was to model the web as a random directed graph, and to imagine a \say{random surfer} who would randomly click on links (represented as directed edges from one site to the next).
As this surfer traversed the web, she would be more likely to arrive at pages which had more inbound links; these pages were \textit{preferred}.
Links from preferred pages are worth more than links from peripheral pages, as the popularity of the preferred page meant it was more likely that a surfer would be travel elsewhere via that page.
Brin and Page's paper describes PageRank as follows:

\begin{quotation}
	\textit{We assume page A has pages T1...Tn which point to it (i.e., are citations).
	The parameter d is a damping factor which can be set between 0 and 1.
	We usually set d to 0.85.
	There are more details about d in the next section.
	Also C(A) is defined as the number of links going out of page A.
	The PageRank of a page A is given as follows:}
	
	\bigskip
		
	PR(A) = (1-d) + d (PR(T1)/C(T1) + ... + PR(Tn)/C(Tn))
	
	\bigskip
	
	\textit{Note that the PageRanks form a probability distribution over web pages, so the sum of all web pages’ PageRanks will be one.
	PageRank or PR(A) can be calculated using a simple iterative algorithm, and corresponds to the principal eigenvector of the normalized link matrix of the web.}
\end{quotation}

The principle difference between the web graph of Sergey and Brin and the socrata graphs we consider here is the variable value associated with the edges.
On the web, a link is a link; there is no notion of a link being worth \say{more} or \say{less}, apart from the page the link originated from.

In our case, edges have weight independent of the popularity of their corresponding nodes.
We can naturally interpret this weight as the strength of the relative preference, and so should seek to allocate preference mass according to the strength of these preferences.
Using the notation of Brin and Page, we introduce a new term $W(A, T_i) \in \mathbb{R}^+$, corresponding to the (normalized) weight of the edge from $A$ to $T_i$:

\[
W(A, T_i) \triangleq \frac{w(A, T_i)}{\sum_{j=1}^n w(A, T_j)}
\]

Now, we present a modified PageRank, which we call \textbf{PrefRank}:

\[
PR(A) \leftarrow (1-d) + d \sum_{i=1}^n \frac{PR(T_i) \times W(A,T_i)}{C(T_i)}
\]

We use the right arrow instead of the equals sign to emphasize that this expression is an \textit{assignment}, updating the values associated with each node at each iteration.
The normalization is important to ensure that the sum of the $PR$ scores remains constant over time.


\subsection{Preference Structuring from Global Relations}

The complexity of the algorithms discussed above are all, in some form, $O(f(n))$.
Finding some way of reducing $n$ would allow all of these techniques to be applied more quickly.

One way to reduce $n$ would be to identify elements of $A$ which are closely related, and to collapse them into \say{super answers} which can be treated as though they were single nodes in a graph.