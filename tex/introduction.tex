This thesis will explore \textit{pairwise preference}, a concept with applications in political science, economics, social science, and machine learning (\cite{jordan}, \cite{arrow}).
A ``pairwise preference'' is simply a preference for one item over another, given two items.

We will argue that such a representation is capable of capturing human subjectivity accurately and precisely.
We will also show that this representation is highly general, and that problems posed in this framework are amenable to many kinds of analysis.

\bigskip

The choice to explore this representation was motivated by a particular reading of the history of science.
In the \textbf{theoretical context} section, we will review aspects of this history and attempt to discern some key themes.
This section will motivate our investigation of pairwise preferences and predict some desirable theoretical properties.

In the \textbf{mechanics} section, we will present the basic elements of preference graphs.
We will demonstrate methods of analysis, drawing on tools from probability and linear algebra.
We will show how a large class of preference resolution problems can be set up within this general framework.

In \textbf{applications}, we will develop various algorithms for analyzing these types of graphs, and discuss their strengths and limitations.

In \textbf{future directions}, we will identify additional avenues of exploration. There are particularly interesting possibilities involving blockchain-based virtual machine (BBVM) technologies like Ethereum.